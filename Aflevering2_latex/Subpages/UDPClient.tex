\chapter{UDP Client}

\section{Opgaveformulering}

Skriv en UDP-client kørende i en anden laptop eller virtuel maskine, som kan sende
et kommando i form af et bogstav ”U”, ”u”, ”L”, ”l” som indtastes af operatøren til
UDP-serveren. Når svaret fra UDP-serveren (beskrevet i punkt 1) modtages,
udskrives dette svar til UDP-client’s bruger. 


\section{UDP Client}

File\_client for UDP-protokollen kræver ikke at IP-adresse sendes direkte fra kommandolinjen, så som et designvalg, startes klienten bare og beder herpå om en IP-adresse som overførslen skal foregå fra. 

Opsætningen af socket og serveradresse foregår derfra som tidligere. 

Brugeren bedes derfra om at taste karakteren for den ønskede information der skal overføres. 

Den indlæste karakter lægges i et char-array, som sendes med funktionen \textit{sendto}:

\begin{lstlisting}
	sendto(clientsocket,msg,BUFFLENGTH,0,(struct sockaddr*)&serv_addr, serv_length);
\end{lstlisting}

Herpå afventes serverens svar ved funktionen \textit{recvfrom}:

\begin{lstlisting}
	recvfrom(clientsocket,msg,BUFFLENGTH,0,(struct sockaddr*) &serv_addr, &serv_length);
\end{lstlisting}

Herpå udskrives den modtagne besked og klienten lukker.


Koden for UDP-klienten er vedlagt