\chapter{Socket programmering - TCP server}

\section{Opgaveformulering}

Der skal udvikles en server med support for en client ad gangen, som kan modtage
en tekststreng fra en client. Serveren skal køre i en virtuel Linux-maskine.
Tekststrengen skal indeholde et filnavn, eventuel ledsaget af en stiangivelse.
Tilsammen skal informationen i tekststrengen udpege en fil af en vilkårlig
type/størrelse i serveren, som en tilsluttet client ønsker at hente fra serveren. Hvis
filen ikke findes skal serveren returnere en fejlmelding til client’en. Hvis filen findes
skal den overføres fra server til client i segmenter på 1000 bytes ad gangen – indtil
filen er overført fuldstændigt. Serverens portnummer skal være 9000. Serverapplikationen
skal kunne startes fra en terminal med kommandoen:
\#./file\_server (for C/C++ applikationers vedkommende)
\#./file\_server.exe (for C\# applikationers vedkommende)
\#python file\_server.py (for Python applikationers vedkommende)
Serveren skal være iterativ, dvs. den skal ikke lukke ned når den har sendt en fil til en
client. Den skal, efter endt filoverførsel, kunne håndtere en ny forespørgsel fra en
client (samme client eller en anden client).
Serveren skal kun kunne håndtere en client ad gangen.