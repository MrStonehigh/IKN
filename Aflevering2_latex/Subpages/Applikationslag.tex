\chapter{Applikationslag}

\section{Opgaveformulering}

Der skal designes, implementeres og testes to applikationer, en client og en server.

Førstnævnte kan anbringes i H1, sidstnævnte kan anbringes i H2.
Den ene applikation (client) skal meddele serveren hvilken fil (evt. incl. sti-angivelse) der skal hentes.
Den anden applikation (server) skal læse og sende filen til klienten i pakkestørrelser på 1000 bytes payload ad gangen.

Client skal modtage disse pakker og gemme dem i en fil.
Såvel server som client er nemme at realisere, idet disse ”applikationslagsapplikationer” allerede er udviklet i øvelse 8 (TCP-baseret client/server).

I øvelse 8 blev der vha. anvendelse af socket-API udført kald som etablerede en connection, overførte filnavn, filstørrelse og fil mellem client og server. Disse kald skal i denne øvelse udskiftes med kald til et transportlaget i en protocol-stack, som I selv udvikler. 

Client og server kan også have samme brugerflade som client-/server-applikationerne i øvelse 8 (TCP/IP-baseret filoverførsel).

\section{File client}
 Vi laver en instants af vores Transportlag klasse, og kalder så vores client receive funktion. Her starter vi med at sende den ønskede filsti til serveren, hvorefter vi modtager størrelsen på filen fra serveren. \\ Herefter starter en løkke, hvor vi modtager 1000bytes ad gangen fra transportlaget og gemmer dem i en fil i rootmappen på clienten.

\section{File server}

Vi har genbrugt en del af applikationslaget fra øvelse 8. I stedet for at lave en socket, som vi gjorde i øvelse 8 Starter vi med at lave en instants af vores Transportlag klasse. Dette gøres med følgende kode: 

\begin{lstlisting}
Transport::Transport transport(BUFSIZE);
\end{lstlisting}

Herefter kalder vi transport.recieve for at få vores filnavn sendt fra clienten, og tester ved hjælp af en if-statement om vi modtager noget fra clienten. 

Vi tjekker om filen eksisterer ved hjælp af følgende kode:

\begin{lstlisting}
long fileSize = check_File_Exists(fileName);
\end{lstlisting}

Efterfølgende kalder vi vores send-funktion i applikationslaget, som tager argumenterne filnavn, filstørrelsen og reference til transportlaget. 
I sendFile() starter vi med at sende størrelsen af filen. Herefter åbner vores fil med ifstream() ligesom i øvelse 8 og sender filen 1000 bytes ad gangen. 


 




 

