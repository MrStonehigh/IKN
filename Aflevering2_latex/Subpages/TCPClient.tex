\chapter{TCP client}

\section{Opgaveformulering}

Der skal udvikles en client kørende i en anden virtuel Linux-maskine. Denne client
skal kunne hente en fil fra den ovenfor beskrevne server. Client’en sender
indledningsvis en tekststreng, som er indtastet af operatøren, til serveren.
Tekststrengen skal indeholde et filnavn + en eventuel stiangivelse til en fil i serveren.
Client’en skal modtage den ønskede fil fejlfrit fra serveren – eller udskrive en
fejlmelding hvis filen ikke findes i serveren. Client-applikationen skal kunne startes fra
en terminal med kommandoen:\\ \\
\#./file\_client <file\_server’s ip-adr.> <[sti] + filnavn> (for C/C++ applikationers
vedkommende)\\
\#./file\_client.exe <file\_server’s ip-adr.> <[sti] + filnavn> (for C\# applikationers
vedkommende)\\ 
\#python file\_client.py <file\_server’s ip-adr.> <[sti] + filnavn> (for Python applikationers
vedkommende)\\

\section{TCP client}

TCP-serveren forventer at clienten sender et eventuelt stinavn, samt navnet på filen som ønskes overført. 
For at kunne eksekvere filen, kræves det ligeledes at der sendes en ip-adresse som der ønskes at hente filen fra. 

Syntaks for eksekvering af beskeden bliver dermed som ønsket:\\
\textit{./file\_server <IP-ADRESSE><FIL-NAVN>} \\

Da alle tre kommandoer tæller som et argument, sørger klienten allerførst for at verificere at der er et korrekt antal argumente:

\begin{lstlisting}
	if(argc>3)
	{
	cout<<argv[1]<<argv[2]<<endl;
	error("Invalid amount of arguments");
	}
\end{lstlisting}


Herefter opretter vi socket, laver opsætning og giver en port, som igen naturligvis er 9000. 

For at benytte IP-adressen som skrives i kommandovinduet, indlæses anden plads i arrayet til serveradressen:

\begin{lstlisting}
server = gethostbyname(argv[1]);
if(server == NULL)
{
error("Host does not exist");
}

\end{lstlisting}

Herpå forbindes der til serveren, med dertil indbygget fejlsikring. Hvis der ikke kan oprettes forbindelse, lukkes clienten:

\begin{lstlisting}
	if(connect(clientsocket,(struct sockaddr *) &serv_addr, sizeof(serv_addr))==-1)
	{
	error("Fejl ved forbindelse til server");
	}
\end{lstlisting}

Herfra anmodes der om at modtage en fil vha \textit{recieveFile()}.

Da klienten skal anmode om en fil, og derfra modtage filen i 1000 bytes af gangen, er processen inddelt i flere dele.

\begin{itemize}
	\item Spørg serveren om filen findes
	\item Hvis filen findes, åbnes en fil som den overførte data kan skrives i
	\item Serveren fortæller hvor stor en fil der skal overføres
	\item Størrelsen af filen bruges som reference ift. den overførte mængde data, som nu kan loopes i mængder af 1000 bytes af gangen. 
	\item Ved endt transmission, lukkes filen
\end{itemize}

Når først filen er overført, termineres TCP-forbindelsen. 

For at udføre første ovennævnte punkt, uddnyttes udleverede LIB-funktion \textit{writeTextTCP}. Den ansøger om et filnavn, som sendes igennem socket til server. 
Herefter kan et svar ventes, som læses med \textit{getFileSizeTCP}. Denne returnerer værdien af filstørrelsen. 

Herpå benyttes standard read/write funktioner til at læse dataen og skrive til en fil. 
Returværdien fra \textit{read} benyttes dog til at holde styr på hvor mange bytes der reelt set er sendt. 

Koden for TCP-klienten er vedlagt. 
