\chapter{UDP Server}

\section{Opgaveformulering}

Skriv en iterativ UDP-server med support for en client ad gangen, som kan modtage
en forespørgsel fra en client.
Forespørgslen fra client til server kan være en af to muligheder: ”U” eller ”L”. Om
bogstaverne er lower case eller upper case skal være ligegyldigt.
\begin{itemize}
	\item Hvis serveren modtager et ”U” skal informationen i filen /proc/uptime
	returneres til client. Denne fil indeholder aktuel information om den samlede
	tid serveren har været kørende siden opstart.\\
	
	\item Hvis serveren modtager et ”L” skal informationen i filen /proc/loadavg
	returneres til client. Denne fil indeholder aktuel information om serverens
	CPU-load.
\end{itemize}

\section{UDP Server}
 
UDP er en protokol som ikke giver garanti for at data kommer frem dvs. den er ustabil. UDP benytter ikke ordnet levering dvs. pakkerne ankommer ikke nødvendigvis i den rigtige rækkefølge til modtageren. 

Ligesom vi gjorde ved TCP starter vi med at sætte en socket op. Vi benytter stadigvæk portnummer 9000 og binder dette til socket'en. 

Vi venter indtil vi modtager en besked fra clienten. 

Herefter laver vi en stream, som vi bruger til at åbne vores fil.

Hvis vores besked indeholder U eller u åbner vi filen /proc/uptime hvorimod hvis beskeden indeholder L eller l åbner vi filen /proc/loadavg. Dette gøres ved at benytte en switch. 


