\chapter{TCP client}

\section{Opgaveformulering}

Der skal udvikles en client kørende i en anden virtuel Linux-maskine. Denne client
skal kunne hente en fil fra den ovenfor beskrevne server. Client’en sender
indledningsvis en tekststreng, som er indtastet af operatøren, til serveren.
Tekststrengen skal indeholde et filnavn + en eventuel stiangivelse til en fil i serveren.
Client’en skal modtage den ønskede fil fejlfrit fra serveren – eller udskrive en
fejlmelding hvis filen ikke findes i serveren. Client-applikationen skal kunne startes fra
en terminal med kommandoen:\\ \\
\#./file\_client <file\_server’s ip-adr.> <[sti] + filnavn> (for C/C++ applikationers
vedkommende)\\
\#./file\_client.exe <file\_server’s ip-adr.> <[sti] + filnavn> (for C\# applikationers
vedkommende)\\ 
\#python file\_client.py <file\_server’s ip-adr.> <[sti] + filnavn> (for Python applikationers
vedkommende)\\

\section{TCP client}
